% Created 2014-09-12 Fri 16:25
\documentclass[11pt]{article}
\usepackage[utf8]{inputenc}
\usepackage[T1]{fontenc}
\usepackage{fixltx2e}
\usepackage{graphicx}
\usepackage{longtable}
\usepackage{float}
\usepackage{wrapfig}
\usepackage{rotating}
\usepackage[normalem]{ulem}
\usepackage{amsmath}
\usepackage{textcomp}
\usepackage{marvosym}
\usepackage{wasysym}
\usepackage{amssymb}
\usepackage{hyperref}
\tolerance=1000
\author{raman}
\date{\textit{<2014-08-19 Tue>}}
\title{Emacspeak At Twenty: Looking Back, Looking Forward}
\hypersetup{
  pdfkeywords={},
  pdfsubject={},
  pdfcreator={Emacs 24.4.50.1 (Org mode 8.2.6)}}
\begin{document}

\maketitle
\tableofcontents

\section{Introduction}
\label{sec-1}

One afternoon in the third week of September 1994, I started
writing myself a small Emacs extension using Lisp Advice to make
Emacs speak to me so I could use a Linux laptop. As Emacspeak
turns twenty, this article is both a quick look back over the
twenty years of lessons learned, as well as a glimpse into what
might be possible as we evolve to a world of connected,
ubiquitous computing. This article draws on \href{http://norvig.com/21-days.html}{Learning To Program
In 10 Years} by Peter Norvig for some of its inspiration.

\section{The Beginning: Using UNIX With Speech Output In 1994}
\label{sec-2}

As a graduate student at \href{http://www.cs.cornell.edu/info/people/raman/raman.html}{Cornell},  I accessed my Unix workstation
(SunOS) from a 486 PC running IBM Screen-Reader.  There was no
means of directly using a UNIX box at the time; after graduating,
I continued doing the same for about six months at Digital
Research in Cambridge   —   the only difference being that my
desktop workstation was now a DEC-Alpha. Throughout this time,
Emacs was my environment of choice for everything from software
development and Internet access to writing documents.


In fall of 1994, I wanted to start using a laptop running Linux;
a colleague (Dave Wecker) was retiring his 386mhz laptop that
already had Linux on it and I decided to inherit it. But there
was only one problem   —   until then I had always accessed a UNIX
machine from a secondary PC running a screen-reader   —   something
that would clearly make no sense with a laptop!

A few weeks earlier, another colleague Win Treese, had pointed
out the interesting possibilities presented by package \texttt{Advice}
 in Emacs 19.23   —   he had sent around a
small snippet of code that magically modified Emacs'
version-control primitive to first create an \emph{RCS} directory if
none existed before adding a file to version control. When I
speculated about using the Linux laptop, Dave remarked   —   you
live in Emacs anyway   —   why dont you just make it talk!

Connecting the dots, I decided to write myself a tool that
augmented Emacs' default behavior to \emph{speak}   —   within about 4
hours, version 0.01 of Emacspeak was up and running.


\section{Key Enabler: Emacs And Lisp Advice}
\label{sec-3}

It took me a couple of weeks to fully recognize the potential of
what I had built with Emacs Lisp Advice. Until then, I had used
screen-readers to listen to the contents of the visual display
  —   but Lisp Advice let me do a lot more   —   it enabled Emacspeak
to generate highly context-specific spoken feedback, augmented by
a set of auditory icons. I later formalized this design under the
name \href{http://en.wikipedia.org/wiki/Self-voicing}{speech-enabled applications}. For a detailed overview of the
architecture of Emacspeak, see the \href{http://emacspeak.sourceforge.net/raman/publications/bc-emacspeak/publish-emacspeak-bc.html}{chapter on Emacspeak} in the
book \href{http://emacspeak.blogspot.com/2007/07/emacspeak-and-beautiful-code.html}{Beautiful Code} from O'Reilly.

\section{Key Component: Text To Speech (TTS)}
\label{sec-4}

Emacspeak is a speech-subsystem for Emacs; it depends on an
external Text-To-Speech (TTS) engine to produce speech. In 1994,
Digital Equipment released what would turn out to be the last in
the line of hardware DECTalk synthesizers, the DECTalk
Express. This was essentially an Intel 386with 1mb of flash
memory that ran a version of the DECTalk TTS software   —   to
date, it still remains my favorite Text-To-Speech engine.
At the time,  I also had a software version of the same engine
running on my DEC-Alpha workstation;  the desire to use either a
software or hardware solution to produce speech output defined
the Emacspeak speech-server architecture.

I went to IBM Research in 1999; this coincided with IBM releasing
a version of the Eloquennce TTS engine on Linux under the name
\emph{ViaVoice Outloud}. My colleague Jeffrey Sorenson implemented an
early version of the Emacspeak speech-server for this engine using
the OSS API; I later updated it to use the ALSA library while on
a flight back to SFO from Boston in 2001. That is still the TTS
engine that is speaking as I type this article on my laptop. 

20 years on, TTS continues to be the weakest link on Linux; the
best available solution in terms of quality continues to be the
Linux port of Eloquence TTS available from Voxin in Europe for a
small price. Looking back across 20 years, the state of TTS on
Linux in particular and across all platforms in general continues
to be a disappointment; most of today's newer TTS engines are
geared toward mainstream use-cases where \emph{naturalness} of the
voice tends to supersede intelligibility at higher
speech-rates. Ironically, modern TTS engines also give
applications far less control over the generated output   —   as a
case in point, I implemented \href{http://www.cs.cornell.edu/home/raman/aster/demo.html}{Audio System For Technical Readings
(AsTeR)} in 1994 using the DECTalk; 20 years later, we implemented
\href{http://allthingsd.com/20130604/t-v-ramans-audio-deja-vu-from-google-a-math-reading-system-for-the-web/}{MathML support} in \href{http://www.chromevox.com/}{ChromeVox}  using Google TTS. In 2013, it turned
out to be difficult or impossible to implement the type of audio
renderings that were possible with the admittedly less-natural
sounding DECTalk!

\section{Emacspeak And Software Development}
\label{sec-5}



Version 0.01 of Emacspeak was written using IBM Screen-Reader on
a PC with a terminal emulator accessing a UNIX  workstation. But
in about 2 weeks, Emacspeak was already a better  environment for
developing Emacspeak in particular and software development in
general.  Here are a few highlights  in 1994 that made Emacspeak
a good software development environment, present-day users of
Emacspeak will see that that was just scratching the surface.

\begin{itemize}
\item Audio formatting using voice-lock to provide aural syntax
highlighting.
\item Succinct auditory icons to provide efficient feedback.
\item Emacs' ability to navigate code structurally   —
\end{itemize}
as opposed to
  moving around by plain-text units such as characters, lines
  and words. S-Expressions are a major win!
\begin{itemize}
\item Emacs' ability to specialize behavior based on major and
minor modes.
\item Ability to browse program code using  tags, and getting
fluent spoken feedback.
\item Completion \textbf{everywhere}.
\item Everything is searchable   —   this is a huge win when you
cannot see the screen.
\item Interactive spell-checking using ISpell with continuous
spoken feedback augmented by  aural highlights.
\item Running code compilation  and being able to jump to errors
with spoken feedback.
\item Ability to move through diff chunks when working with source
code and source control systems;  refined diffs as provided
by the \uline{ediff} package when speech-enabled is a major
productivity win.
\item Ability to easily move between email, document authoring and
programming   —   though this may appear trivial, it continues
to be one of Emacs' biggest wins.
\end{itemize}


Long-term Emacs users will recognize all of the above as  being
among the reasons why they do most things inside Emacs   —   there
is little that is Emacspeak specific in the above list   —   except
that Emacspeak was able to provide fluent, well-integrated
contextual feedback for all of these tasks. And that was a
game-changer given what I had had before Emacspeak. As a case in
point, I did not dare program in Python before  I speech-enabled
Emacs' Python-Mode; the fact that white space is significant in
Python made it difficult to program using a plain screen-reader
that was unaware of the semantics of the underlying content being accessed.

\section{Emacspeak And Authoring Documents}
\label{sec-6}

In 1994, my preferred environment for authoring \textbf{all} documents
was \uline{\LaTeX{}} using the Auctex package. Later I started writing
either \LaTeX{} or HTML using the appropriate support modes; today I
use \uline{org-mode} to do most of my content authoring. Personally, I
have never been a fan of What You See Is What You Get (WYSIWYG )
authoring tools   —   in my experience that places an undue burden
on the author by  drawing attention away from the content
to focus on the final appearance. An added benefit  of creating
content in Emacs in the form of light-weight markup  is that the
content is long-lived   —   I can still usefully process and re-use
things I have written 25 years ago.

Emacs, with Emacspeak providing audio formatting and
context-specific feedback remains my environment of choice for
writing all forms of content ranging from simple email messages
to polished documents for print publishing. And it is worth
repeating that I \textbf{never} need to focus on what the content is
going to look like   —   that job is best left to the computer.

As an example of producing high-fidelity visual content, see
this write-up on \href{http://emacspeak.sourceforge.net/raman/publications/polyhedra/}{Polyhedral Geometry} that  I published in 2000;
all of the content, including the drawings were created by me
using Emacs. 

\section{Emacspeak And The Early Days Of The Web}
\label{sec-7}

Right around the time that I was writing version 0.01 of
emacspeak, a far more significant software movement was under way
  —   the World Wide Web was moving from the realms of academia to
the mainstream world with the launch of NCSA Mosaic   — 
  and in
late 1994 by the first commercial Web browser in Netscape
Navigator. Emacs had always enabled integrated access to FTP
archives via package \emph{ange-ftp}; in late 1993, William Perry
released Emacs-W3, a Web browser for Emacs written entirely in
Emacs Lisp. W3 was one of the first large packages to be
speech-enabled by Emacspeak   —   later it was the browser on which
I implemented the first draft of the \href{http://www.w3.org/TR/CSS2/aural.html}{Aural CSS
specification}. Emacs-W3 enabled many early innovations in the
context of providing non-visual access to Web content, including
audio formatting and structured content navigation; in summer of
1995, Dave Raggett and I outlined a few extensions to HTML Forms,
including the \uline{label} element as a means of associating metadata
with interactive form controls in HTML, and many of these ideas
were prototyped in Emacs-W3 at the time. Over the years, Emacs-W3 fell
behind the times   —   especially as the Web moved away from
cleanly structured HTML to a massive soup of unmatched tags. This
made parsing and error-correcting badly-formed HTML markup
expensive to do in Emacs-Lisp  —  
and performance suffered. To add
to this, mainstream users moved away because Emacs' rendering
engine at the time was not rich enough to provide the type of
visual renderings that users had come to expect. The advent of
DHTML, and JavaScript based Web Applications finally killed off
Emacs-W3 as far as most Emacs users were concerned.

But Emacs-W3 went through a revival on the emacspeak audio
desktop in late 1999 with the arrival of XSLT, and Daniel
Veillard's excellent implementation via the \uline{libxml2} and
\uline{libxslt} packages. With these in hand, Emacspeak was able to
hand-off the bulk of HTML error correction to the \uline{xsltproc}
tool. The lack of visual fidelity didn't matter much for an
eyes-free environment; so Emacs-W3 continued to be a useful tool
for consuming large amounts of Web content that did not require
JavaScript support.

During the last 24 months, \uline{libxml2} has been built into Emacs;
this means that you can now parse arbitrary HTML as found in the
wild without incurring  a performance hit. This functionality was
leveraged first by package \uline{shr} (Simple HTML Renderer) within
the \uline{gnus} package for rendering HTML email; later, the author of
\uline{gnus} and \uline{shr} created a new light-weight HTML viewer called
\uline{eww} that is now part of Emacs 24. the improved support for
variable pitch fonts and image embedding, Emacs is once again
able to provide visual renderings for a large proportion of
text-heavy Web content that is useful for mainstream Emacs users
to view at least some Web content within Emacs; during the last
year, I have added support within emacspeak to \href{http://emacspeak.blogspot.com/2014/05/emacspeak-eww-updates-for-complete.html}{extend package
\uline{eww}} with support for DOM filtering and quick content
navigation.

\section{Audio Formatting: Generalizing Aural CSS}
\label{sec-8}


A key idea in Audio System For Technical Readings \href{http://www.cs.cornell.edu/home/raman/aster/aster-toplevel.html}{(AsTeR)} was the
use of various voice properties in combination with non-speech
auditory icons to create rich aural renderings. When I
implemented Emacspeak, I brought over the notion of audio
formatting to all buffers in Emacs by creating a \uline{voice$_{\text{lock}}$}
module that paralleled Emacs' \uline{font$_{\text{lock}}$} module. The visual
medium is far richer in terms of available fonts and colors as
compared to voice parameters available on TTS engines  — 
consequently, it did not make sense to directly map Emacs' \uline{face}
properties to voice parameters. To aid in projecting visual
formatting onto auditory space, I created property \uline{personality}
analogous to Emacs' \uline{face} property that could be applied to
content displayed in Emacs; module \uline{voice$_{\text{lock}}$} applied that
property appropriately, and the Emacspeak core handled the
details of mapping personality values to the underlying TTS
engine. 

The values used in property \uline{personality} were abstract, i.e.,
they were independent of any given speech engine. Later in the
fall of 1995, I re-expressed these set of abstract voice
properties in terms of Aural CSS; the work was published as a
first draft toward the end of 1995, and implemented in Emacs-W3
in early 1996. Aural CSS was an appendix in the CSS-1.0
specification; later, it graduated to being its own module within
CSS-2.0.

Later in 1996, all of Emacs' \uline{voice-lock} functionality was
re-implemented in terms of Aural CSS; the implementation has
stood the test of time in that as I added support for more TTS
engines, I was  able to implement engine-specific mappings of
Aural-CSS values. This meant that the rest of Emacspeak could
define various types of voices for use in specific contexts
without having to worry about individual TTS engines.
Conceptually, property \uline{personality} can be thought of as holding
an \uline{aural display list}  —   various parts of the system can
annotate pieces of text with relevant properties that finally get
rendered in the aggregate. 
This model also works well with the notion of Emacs overlays
where a moving overlay is used to temporarily highlight text that
has other context-specific properties applied to it.


Audio formatting as implemented in Emacspeak is extremely
effective when working with all types of content ranging from
richly structured mark-up documents (\LaTeX{}, org-mode) and
formatted Web pages to program source code. Perceptually,
switching to audio formatted output feels like switching from a
black-and-white monitor to a rich color display. Today,
Emacspeak's audio formatted output is the only way I can
correctly write \uline{else if} vs \uline{elsif} in various programming
languages!

\section{Conversational Gestures For The Audio Desktop}
\label{sec-9}

By 1996, Emacspeak was the only piece of adaptive technology I
used; in fall of 1995, I had moved to Adobe Systems from DEC
Research to focus on enhancing the Portable Document Format (PDF)
to make PDF content repurposable. Between 1996 and 1998, I was
primarily focused on electronic document formats  —   I took this
opportunity to step back and evaluate what I had built as an
auditory interface within Emacspeak. This retrospect proved
extremely useful in gaining a sense of perspective and led to
formalizing the high-level concept of \emph{Conversational Gestures}
and structured browsing/searching as a means of thinking about user interfaces.

By now, Emacspeak was a complete environment — I formalized what
it provided under the moniker \emph{Complete Audio Desktop}. The fully
integrated user experience allowed me to move forward with
respect to defining interaction models that were highly optimized
to eyes-free interaction — as an example, see how Emacspeak
interfaces with modes like \uline{dired} (Directory Editor) for
browsing and manipulating the filesystem, or \uline{proced} (Process Editor) for
browsing and manipulating running processes. Emacs' integration
with \uline{ispell} for spell checking, as well as its various
completion facilities ranging from minibuffer completion to other
forms of dynamic completion while typing text provided more
opportunities for creating innovative forms of eyes-free
interaction. With respect to what had gone before (and is still
par for the course as far as traditional screen-readers are
concerned), these types of highly dynamic interfaces present a
challenge. For example, consider handling a completion interface
using a screen-reader that is speaking the visual display. There
is a significant challenge in deciding \emph{what to speak} e.g., when
presented with a list of completions, the currently typed text,
and the default completion, which of these should you speak, and
in what order?
The problem gets harder when you consider that the underlying
semantics of these items is generally not available from
examining the visual presentation in a consistent manner. By
having direct access to the underlying information being
presented, Emacspeak had a leg up with respect to addressing the
higher-level question  —   when you do have access to this
information, how to you present it effectively in an eyes-free
environment? For this and many other cases of dynamic
interaction, a combination of audio formatting, auditory icons,
and the ability to synthesize succinct messages from a
combination of information items  —   rather than having to
forcibly speak each item as it is rendered visually provided for
highly efficient eyes-free interaction. 


This was also when I stepped back to build out Emacspeak's table
browsing facilities — see the online Emacspeak documentation for
 details on Emacspeak's table  browsing functionality which
continues to remain one of the richest  collection of  end-user
affordances for working with two-dimensional data.

\subsection{Speech-Enabling Interactive Games}
\label{sec-9-1}

So in 1997, I went the
next step in asking  —   given access to the underlying
infromation, is it possible to build effective eyes-free
interaction to highly interactive tasks? I picked \uline{Tetris} as a
means of exploring this space, the result was an Emacspeak
extension to speech-enable module \uline{tetris.el}. The details of
what was learned were published as a paper in Assets 98, and
expanded as a chapter on Conversational Gestures in my book on
Auditory Interfaces; that book was in a sense a culmination of
stepping back and gaining a sense of perspective of what I had
build during this period. The work on Conversational Gestures
also helped in formalizing the abstract user interface layer that
formed part of the \href{http://www.w3.org/MarkUp/Forms/}{XForms}  work at the W3C.

Speech-enabling games for effective eyes-free interaction  has
proven highly educational. Interactive games are typically built to challenge
the user,  and if the eyes-free interface is inefficient,  you
just wont play the game —
 contrast this with a task that you \textbf{must} perform, where you're
likely to make do with a sub-optimal interface.  Over the years,
Emacspeak has come to include eyes-free interfaces to several
games including Tetris, SuDoKu, and of late the popular
2048-game. Each of these have in turn contributed to  enhancing
the interaction model in Emacspeak, and those innovations
typically make their way to the rest of the environment. 


\section{Accessing Media Streams}
\label{sec-10}

Streaming real-time audio on the Internet became a reality with
the advent of RealAudio in 1995; soon there were a large number
of media streams available on the Internet ranging from music
streams to live radio stations. But there was an interesting
twist — for the most part, all of these media streams expected
one to look at the screen, even though the primary content was
purely audio (streaming video hadn't arrived yet!). Starting in
1996, Emacspeak started including a variety of eyes-free
front-ends for accessing media streams. Initially, this was
achieved by building a wrapper around \uline{trplayer} — a headless
version of RealPlayer; later I built Emacspeak module
\uline{emacspeak-m-player} for interfacing with package \uline{mplayer}. A
key aspect of streaming media integration in emacspeak is that
one can launch and control streams without ever switching away
from one's primary task; thus, you can continue to type email or
edit code while seamlessly launching and controlling media
streams. Over the years, Emacspeak has come to integrate with
Emacs packages like \uline{emms} as well as providing wrappers for
\uline{mplayer} and \uline{alsaplayer} — collectively, these let you
efficiently launch all types of media streams, including
streaming video, without having to explicitly switch context.


In the mid-90's, Emacspeak started including a directory of media
links to some of the more popular radio stations — primarily as a
means of helping users getting started — Emacs' ability to
rapidly complete directory and file-names turned out to be the
most effective means of quickly launching everything from
streaming radio stations to audio books. And even beter — as the
Emacs community develops better and smarter ways of navigating
the filesystem using completions, e.g., package \uline{ido}, these
types of actions become even more efficient!

\section{EBooks: Bookshare, Calibre And Epub: Ubiquitous Access To Books}
\label{sec-11}


AsTeR — was motivated by the increasing availability of technical
material as online electronic documents. While AsTeR processed
the \TeX{} family of markup languages, more general ebooks came in a
wide range of formats, ranging from plain text generated from
various underlying file formats to structured EBooks, with
Project \href{http://www.gutenberg.org/}{Gutenberg} leading the way. During the mid-90's, I had
access to a wide range of electronic materials from sources such
as O'Reilly Publishing and various electronic journals -- The
Perl Journal (TPJ) is one that I still remember fondly. 

Emacspeak provided fairly light-weight but efficient access to
all of the electronic books I had on my local disk — Emacs'
strengths with respect to browsing textual documents meant that I
needed to build little that was specific to Emacspeak. The late
90's saw the arival of Daisy as an XML-based format for
accessible electronic books. The last decade has seen the rapid
convergence to \textbf{epub} as a distribution format of choice for
electronic books. Emacspeak provides interaction modes that make
organizing, searching and reading these materials on the
Emacspeak Audio Desktop a pleasant experience. Emacspeak also
provides an OCR-Mode — this enables one to call out to an
external OCR program and read the content efficiently.

The somewhat informal process used by publishers like O'Reilly to
make technical material available to users with print impairments
was later formalized by \href{https://www.bookshare.org/#mainContent}{BookShare} — today, qualified users can
obtain a large number of books and periodicals initially as
Daisy-3 and increasingly as \uline{EPub}. BookShare provides a RESTful
API for searching and downloading books; Emacspeak module
\uline{emacspeak-bookshare} implements this API to create a client for
browsing the BookShare library, downloading and organizing books
locally, and an integrated ebook  reading mode to round off the
experience.

A useful complement to this suite of tools is the Calibre package
for organizing ones ebook collection; Emacspeak now implements an
\textbf{EPub Interaction} mode that leverages Calibre (actually sqlite3)
to search and browse books, along with an integrated \textbf{EPub mode}
for reading books.

\section{Leveraging Computational Tools: From SQL And R To IPython Notebooks}
\label{sec-12}

The ability to invoke external processes and interface with them
via a simple read-eval-loop (REPL) is perhaps one of Emacs'
strongest extension points. This means that a wide variety of
computational tools become immediately available for embedding
within the Emacs environment — a facility that has been widely
exploited by the Emacs community. Over the years, Emacspeak has
leveraged every one of these facilities to provide a
well-integrated auditory interface.

Starting from a tight code, eval, test form of iterative
programming as encouraged by Lisp and applied to languages like
Python and Ruby to explorative computational tools such as R for
data analysis and SQL for database interaction, the Emacspeak
Audio Desktop has come to encompass a collection of rich tools
that provide an efficient eyes-free experience backed up by
consistent audio formatted output.

\section{Social Web: EMail,Instant Messaging, Blogging  And Tweeting Using Open Protocols}
\label{sec-13}

The ability to process large amounts of email and electronic news
has always been a feature of Emacs. I started using package \uline{vm}
for email in 1990, along with \uline{gnus} for Usenet access many years
before developing Emacspeak. So these were the first major
packages that Emacspeak speech-enabled. Being able to access the
underlying data structures used to visually render email messages
and Usenet articles enabled Emacspeak to produce rich, succinct
auditory output — this vastly increased my ability to consume and
organize large amounts of information. Toward the turn of the
century, instant messaging arived in the mainstream — package
\uline{tnt} provided an Emacs implementation of a chat client that
could communicate with users on the then popular AOL Instant
Messenger platform. At the time, I worked at IBM Research, and
inspired by package \uline{tnt}, I created an Emacs client called
\uline{ChatterBox} using the Lotus Sametime API — this enabled me to
communicate with colleagues at work from the comfort of
Emacs. Packages like \uline{vm}, \uline{gnus}, \uline{tnt} and \uline{ChatterBox} provide
an interesting example of how availability of a clean underlying
API to a specific service or content stream can encourage the
creation of efficient (and different) user interfaces. The
touchstone of such successful implementations is a simple test —
can the user of a specific interface tell if the person whom he
is communicating with is also using the same interface? In each
of the examples enumerated above, a user at one end of the
communication chain cannot tell, and in fact shouldn't be able to
tell what client the user at the other end is using. Contrast
this with closed services that have an inherent \emph{lock-in} model
e.g., proprietary word processors that use undocumented
serialization formats — for a fun read, see this write-up on
\href{http://emacspeak.sourceforge.net/publications/colored-paper.html}{Universe Of Fancy Colored Paper}.


Today, my personal choice for instant messaging is the open
Jabber platform. I connect to Jabber via Emacs package
\uline{emacs-jabber} and with Emacspeak providing a light-weight
wrapper for generating the eyes-free interface, I can communicate
seamlessly with colleagues and friends around the world.

As the Web evolved to encompass ever-increasing swathes of
communication functionality that had already been available on
the Internet, we saw the world move from Usenet groups to \uline{Blogs}
— I remember initially dismissing the blogging phenomenon as just
a re-invention of Usenet in the early days. However, mainstream
users flocked to Blogging, and I later realized that blogging as
a publishing platform brought along interesting features that
made communicating and publishing information \textbf{much} easier. In
2005, I joined Google; during the winter holidays that year, I
implemented a light-weight client for Blogger that became the
start of Emacs package \uline{g-client} — this package provides Emacs
wrappers for Google services that provide a RESTful API.


\section{The RESTful Web:  Web Wizards And URL Templates For Faster Access}
\label{sec-14}

Today, the Web, based on URLs and HTTP-style protocols is widely
recognized as a platform in its own right. This platform emerged
over time — to me, Web APIs arrived in the late 90's when I
observed the following with respect  to my own behavior  on many
popular sites:

\begin{enumerate}
\item I opened a Web page that took a while to load (remember,  I
was still using Emacs-W3),
\item I then searched through the page to find a form-field that
I filled out, e.g. start and end destinations on Yahoo Maps,
\item I hit \uline{submit}, and once again waited for a heavy-weight
HTML page to load,
\item And finally, I hunted through the rendered content to find
what I was looking for.
\end{enumerate}

This pattern repeated across a wide-range of interactive Web
sites ranging from AltaVista for search (this was pre-Google), Yahoo Maps for directions, and Amazon for product searches
to name but a few. So I decided to automate away the pain by
creating Emacspeak module \uline{emacspeak-websearch}
that  did the following: 

\begin{enumerate}
\item Prompt via the minibuffer for the requisite fields,
\item Consed up an HTTP GET URL,
\item Retrieved this URL,
\item And filtered out the specific portion of the HTML  DOM that
held the  generated response.
\end{enumerate}

Notice that the above implementation hard-wires the CGI parameter
names used by a given Web application into the code implemented
in module \uline{emacspeak-websearch}.  REST as a design pattern had
not yet been recognized, leave alone formalized, and module
\uline{emacspeak-websearch} was initially decryed as being fragile.

However, over time, the CGI parameter names remained fixed — the
 only things that have required updating in the Emacspeak
 code-base are the content filtering rules that extract the
 response — for popular services, this has averaged about one to
 two times a year.


I later codified these filtering rules in terms of XPath, and
also integrated XSLT-based pre-processing of incoming HTML
content before it got handed off to Emacs-W3 — and yes,
Emacs/Advice once again came in handy with respect to injecting
XSLT pre-processing into Emacs-W3!

Later, in early 2000, I created companion module
\uline{emacspeak-url-templates} — partially inspired by Emacs'
\uline{webjump} module.
URL templates in Emacspeak leveraged the  recognized REST
interaction pattern to provide a large collection of Web widgets
that could be quickly invoked to provide rapid access to the
right pieces of information on the Web.

The final icing on the cake was the arrival of RSS and Atom feeds
and the consequent deep-linking into content-rich sites — this
meant that Emacspeak could provide audio renderings of useful
content without having to deal with complex visual navigation!
While Google Reader existed, Emacspeak provided a light-weight
\uline{greader} client for managing ones feed subscriptions; with the
demise of Google Reader, I implemented module \uline{emacspeak-feeds}
for organizing feeds on the Emacspeak desktop. A companion
package \uline{emacspeak-webspace} implements additional goodies
including a continuously updating ticker of headlines taken from
the user's collection of subscribed feeds.


\section{Mashing It Up: Leveraging Evolving Web APIs}
\label{sec-15}

The next step in this evolution came with the arrival of richer
Web APIs — especially ones that defined a clean client/server
separation. In this respect, the world of Web APIs is a somewhat
mixed bag in that many Web sites equate a Web API  with a
JS-based API that can be exclusively invoked from within a Web-Browser
run-time. 
The issue with that type of API  binding is that the only runtime
that is supported is a full-blown Web browser; but the arrival of
native mobile apps  has actually proven a net positive in
encouraging sites to create a cleaner separation. Emacspeak has
leveraged these APIs to create Emacspeak front-ends 
to many useful services, here are a few:

\begin{enumerate}
\item Minibuffer completion for Google Search using Google Suggest
to provide completions.
\item Librivox for browsing  and playing free audio books.
\item NPR  for browsing and playing NPR archived programs.
\item BBC for playing a wide variety of streaming content
available from the BBC.
\item A Google Maps front-end that  provides instantaneous access
to directions and Places search.
\item Access to Twitter via package \uline{twittering.-mode}.
\end{enumerate}


And a lot more than will fit this margin! This is an example of
generalizing the concept of a mashup as seen on the Web with
respect to creating hybrid applications by bringing together a
collection of different Web APIs. Another way to think of such
separation is to view an application as a \textbf{head} and a \textbf{body} —
where the \textbf{head} is a specific user interface, with the \textbf{body}
implementing the application logic. A cleanly defined separation
between the \textbf{head} and \textbf{body} allows one to attach \emph{different}
user interfaces i.e., \textbf{heads} to the given \textbf{body} without any
loss of functionality, or the need to re-implement the entire
application. Modern platforms like Android enable such separation
via an \href{http://developer.android.com/reference/android/content/Intent.html}{Intent} mechanism. The Web platform as originally defined
around URLs is actually well-suited to this type of separation —
though the full potential of this design pattern remains to be
fully realized given today's tight association of the Web to the
Web Browser.

\section{Conclusion}
\label{sec-16}


In 1996, I wrote an article entitled \href{http://www.drdobbs.com/user-interface-a-means-to-an-end/184410453}{User Interface — A Means To
An End} pointing out that the size and shape of computers were
determined by the keyboard and display. This is even more true in
today's world of tablets, phablets and large-sized phones — with
the only difference that the keyboard has been replaced by a
touch screen. The next generation in the evolution of \textbf{personal}
devices is that they will become truly personal by being wearables
— this once again forces a separation of the user interface
peripherals from the underlying compute engine. Imagine a variety
of wearables that collectively connect to ones cell phone, which
itself connects to the cloud for all its computational and
information needs. Such an environment is rich in possibilities
for  creating a wide variety of  user experiences to a single
underlying body of information; Eyes-Free interfaces as pioneered
by systems like Emacspeak will come to play an increasingly vital
role alongside  visual interaction when this comes to pass.



--T.V. Raman, San Jose, CA, September 12, 2014






\section{References}
\label{sec-17}

\begin{itemize}
\item \href{http://emacspeak.sourceforge.net/raman/aui/aui.html}{Auditory User Interfaces}::  Klewer Publishing, 1997.
\item[{Advice }] An Emacs Lisp package by    \href{http://www.isi.edu/~hans/}{Hans Chalupsky} that
became part of Emacs 19.23.
\item \href{http://artlung.com/smorgasborg/C_R_Y_P_T_O_N_O_M_I_C_O_N.shtml}{In The Beginning Was The Command Line} By Neal Stephenson
\item[{\href{http://emacspeak.blogspot.com/2007/07/emacspeak-and-beautiful-code.html}{Beautiful Code}}] An overview of the Emacspeak architecture.
\item[{\label{http://emacspeak.sourceforge.net/raman/publications/chi96-emacspeak/-Speech-Enabled-Applications}]]}] Emacspeak at CHI 1996.
\item[{EWW}] Emacspeak  \href{http://emacspeak.blogspot.com/2014/05/emacspeak-eww-updates-for-complete.html}{extends EWW }.
\end{itemize}
% Emacs 24.4.50.1 (Org mode 8.2.6)
\end{document}
